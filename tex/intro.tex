\newpage
\section{Введение}
Целью выпускной квалификационной работы (ВКР) является разработка в системе компьютерной алгебры алгоритма, позволяющего эффективно преобразовывать операторные матрицы к приведённому по строкам виду.


Объект исследования – компьютерная алгебра операторных матриц


Предмет работы: алгоритмы вычисления в системах компьютерной алгебры; программная реализация алгоритма, позволяющего преобразовывать операторные матрицы к приведённому по строкам виду.


В работе определены информационные и методологические основания для исследования, которыми явились труды следующих ученых: Абрамова С.А., Баркату М.А., Чена Г., Лабана Дж., Шторйоханна А. и др.


Основные методы работы: анализ, абстрагирование, формализация, алгоритмизация, моделирование, …., программирование, графического и текстового представления.


Ключевые научно-прикладные положения и предложения, полученные в работе, согласуются с ….


Теоретическая значимость работы определяется: 


Практическая значимость работы состоит в том, что 


Результаты исследования ВКР получили апробацию 


Содержание работы раскрыто в четырех разделах, 4 таблицах и 11 рисунках. Список источников составляет 31 наименование. Основные модельные и программные решения отражены в приложениях А-В 

ffffffffffffffffffffffffffffffffffffffffffffffffffffffffffffffffffffffffffffff

